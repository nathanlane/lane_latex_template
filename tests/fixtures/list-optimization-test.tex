\documentclass[11pt,a4paper]{article}
\usepackage{../../paper/paperstyle}
\usepackage{../../paper/gridoverlay}
\usepackage{lipsum}

\begin{document}

% Enable grid visualization
\showgrid

\section{Refined List Typography Tests}

This document demonstrates the optimized list typography with elegant bullets designed specifically for TeX Gyre Pagella.

\subsection{Standard Itemize Lists}

The refined bullets are now 80\% of their original size with optical adjustments:
\begin{itemize}
\item First item with elegant scaled bullet at 25\% gray
\item Second item showing improved visual weight
\item Third item maintaining perfect grid alignment
  \begin{itemize}
  \item Nested item with refined en-dash at 30\% gray
  \item Another nested item showing hierarchy
    \begin{itemize}
    \item Third level with delicate open circle at 35\% gray
    \item Provides clear visual progression
    \end{itemize}
  \item Back to second level
  \end{itemize}
\item Return to primary level
\end{itemize}

\subsection{Comparison: Old vs New Bullets}

Old style (if using \texttt{\textbackslash bulletmark} directly):
\begin{itemize}
\item[\textcolor{gray}{0.20}{\textbullet}] Original bullet at 20\% gray (80\% black)
\item[\elegantbullet] New elegant bullet - scaled to 80\%, 25\% gray (75\% black)
\item[\elegantdot] Alternative dot style - naturally smaller and more refined
\end{itemize}

\subsection{Numbered Lists with Refinements}

\begin{enumerate}
\item First item with oldstyle figures
\item Second item showing left alignment
\item Items with double digits align properly:
  \setcounter{enumi}{8}
\item Ninth item
\item Tenth item demonstrates reserved space
\item Eleventh item maintains alignment
\end{enumerate}

\subsection{Readable List Variant}

For content requiring more visual breathing room:
\begin{readableitem}
\item First item with increased spacing (0.5 baseline units between items)
\item Second item showing enhanced readability
\item Third item with more generous top/bottom spacing
\item This variant is ideal for:
  \begin{readableitem}
  \item Important points requiring emphasis
  \item Lists within dense academic text
  \item Situations where scanning is important
  \end{readableitem}
\end{readableitem}

\subsection{Academic En-dash Lists}

Following university style guide conventions:
\begin{academicitem}
\item First point with refined en-dash marker
\item Second point showing professional appearance
\item Third point maintaining elegant proportions
\end{academicitem}

\subsection{Mixed Content Integration}

\lipsum[1][1-3]

\begin{itemize}
\item The list integrates seamlessly with surrounding paragraphs
\item Spacing before and after has been refined for better flow
\item Grid alignment is maintained throughout
\end{itemize}

\lipsum[2][1-3]

\subsection{Inline Lists}

For brief enumerations, we can use inline lists: \begin{inlineitem}
\item first \item second \item third
\end{inlineitem}, which flow naturally within the text.

\subsection{Special List Variants}

\subsubsection{Compact Lists}
\begin{compactitem}
\item Minimal spacing
\item Dense information
\item Reference lists
\end{compactitem}

\subsubsection{Display Lists}
\begin{displayitem}
\item \textbf{Key Finding:} Bold items with black bullets
\item \textbf{Major Result:} Generous spacing for emphasis
\item \textbf{Important Note:} Full baseline units between items
\end{displayitem}

\end{document}