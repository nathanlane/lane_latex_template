\documentclass[11pt,a4paper]{article}
\usepackage{../../paper/paperstyle}
\usepackage{../../paper/gridoverlay}  % Enable grid visualization
\usepackage{lipsum}

\begin{document}

% Enable grid visualization
\showgrid

% Test all heading levels with content
\section{Primary Heading Level}
\lipsum[1][1-3]

\subsection{Secondary Heading Level}
\lipsum[2][1-3]

\subsubsection{Tertiary Heading Level}
\lipsum[3][1-3]

\paragraph{Paragraph Heading Level}
\lipsum[4][1-3]

% Test heading sequences
\section{Testing Grid Alignment}
This section tests whether all headings maintain perfect baseline grid alignment.
The optimized spacing should ensure 100\% grid compliance without optical adjustments.

\subsection{Mathematical Verification}
Each heading block has a total height in exact grid units:
\begin{itemize}
\item Section: 79.2pt total (6 grid units)
\item Subsection: 59.4pt total (4.5 grid units)
\item Subsubsection: 39.6pt total (3 grid units)
\item Paragraph: 33pt total (2.5 grid units)
\end{itemize}

\subsubsection{Leading Optimization}
The key improvements:
\begin{itemize}
\item Section: 26.4pt leading (2 units) instead of 22pt
\item Subsection: 19.8pt leading (1.5 units) instead of 17pt
\item No optical adjustments needed
\end{itemize}

\paragraph{Flexible Spacing}
The asymmetric flexibility (plus-only for space after) ensures the grid floor is maintained.

% Test orphan prevention
\section{Orphan Prevention Test}
\lipsum[5-6]

\subsection{Widow Control}
\lipsum[7-8]

% Test consecutive headings
\section{Consecutive Headings}
\subsection{Immediate Subsection}
\subsubsection{And Subsubsection}
\paragraph{With Paragraph}
Content follows here to verify spacing.

\end{document}