\documentclass[11pt,a4paper]{article}
\usepackage{../../paper/paperstyle}
\usepackage{lipsum}

\begin{document}

\section{Emphasis Hierarchy Test Suite}

This document demonstrates the refined emphasis hierarchy optimized for TeX Gyre Pagella, implementing principles from Butterick, Hochuli, and Bringhurst.

\subsection{Basic Emphasis Levels}

\subsubsection{Level 1: Standard Emphasis}

Regular text with \emph{standard emphasis} demonstrates the primary emphasis level. This is the most common form and should comprise 90\% of all emphasis usage. When we have \emph{emphasis within emphasis, it returns to roman} following the contrast principle.

\subsubsection{Level 2: Strong Emphasis}

For critical terms requiring immediate attention, we use \strongemph{strong emphasis}. This should be used sparingly---less than 5\% of total text according to Butterick's guidelines.

\subsubsection{Level 3: Metadata Emphasis}

Author names like \person{Hermann Zapf} and \person{Matthew Butterick} use elegant small caps. Technical acronyms such as \acro{PDF}, \acro{HTML}, and \acro{NASA} receive specialized treatment.

\subsubsection{Level 4: Critical Emphasis}

The highest level, \critical{critical warning}, should be reserved for the most important notices. Use this extremely rarely.

\subsection{Specialized Semantic Commands}

\paragraph{Technical Terminology}
When introducing a \term{baseline grid}, we use the term command. The \term{modular scale} provides mathematical harmony. These terms appear in italic on first use.

\paragraph{Work Titles}
References to published works like \work{The Elements of Typographic Style} or \work{Practical Typography} use standard italics.

\subsection{Nested Emphasis Demonstrations}

\subsubsection{Italic Within Italic}

Consider this passage: \emph{The author writes about \emph{nested emphasis} with clarity}. Note how the inner emphasis returns to roman for contrast.

\subsubsection{Smart Emphasis Commands}

Using \texttt{\textbackslash smartitalic}: Regular text with \smartitalic{smart italic} and \emph{italic text with \smartitalic{smart italic} inside}.

Using \texttt{\textbackslash smartbold}: Regular text with \smartbold{smart bold} and \textbf{bold text with \smartbold{smart bold} inside}.

\subsection{Small Caps Tracking Demonstration}

\paragraph{Size-Responsive Tracking}

Body text small caps: \meta{metadata example}

{\large Large small caps: \meta{metadata example}}

{\Large Display small caps: \meta{metadata example}}

{\LARGE Title small caps: \meta{metadata example}}

\subsection{Weight Relationships}

\begin{description}
\item[Regular (400)] Normal text weight for body content
\item[\strongemph{Bold (700)}] True bold at 1.75× regular weight
\item[\emph{Italic}] True italic with distinct letterforms
\item[\textbf{\emph{Bold Italic}}] Combined weight and style
\item[\meta{Small Caps}] Proportional capitals at 75\% x-height
\item[\critical{Bold Small Caps}] Maximum emphasis level
\end{description}

\subsection{Usage Guidelines Demonstration}

\paragraph{Correct Usage}
\begin{itemize}
\item First mention of \term{kerning} as a technical term
\item Author names: \person{Jost Hochuli} and \person{Robert Bringhurst}
\item Critical warnings use \critical{stop} sparingly
\item Standard \emph{emphasis} for most needs
\end{itemize}

\paragraph{Accessibility Considerations}
All emphasis relies on weight and style, not color alone. The bold weight provides 1.75× the stroke weight of regular text, ensuring sufficient contrast. Small caps maintain legibility at 9pt equivalent or larger.

\subsection{Extended Text Sample}

\lipsum[1]

In this paragraph, we see how \emph{emphasis integrates naturally} into running text. The \person{Zapf} typeface family, including \acro{TEX} Gyre Pagella, demonstrates \term{optical sizes} effectively. When discussing \work{Detail in Typography}, \person{Hochuli} emphasizes that \emph{restraint in the use of \emph{multiple} emphasis styles} creates professional documents.

\lipsum[2]

\end{document}