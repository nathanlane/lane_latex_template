\documentclass[11pt]{article}
\usepackage{../../paper/paperstyle}
\usepackage{lipsum}

\begin{document}

\section{Introduction}

\sectionopening{This demonstrates the minimal first-line small caps treatment,} which provides subtle visual hierarchy without being decorative. The rest of the paragraph continues normally. This is the most conservative enhancement suitable for academic articles.

\lipsum[1]

\section{Standard Academic Writing}

For most academic articles, no special opening treatment is needed. The existing section formatting with proper spacing provides sufficient hierarchy. Additional embellishments should be used only when they serve a clear purpose.

\lipsum[2]

\sectionbreak

\section{After a Section Break}

The section break above uses only white space—no ornaments or rules. This is the recommended approach for academic articles where clarity and professionalism are paramount.

\lipsum[3]

\section{Opening Paragraph Environment}

\begin{openingparagraph}
The opening paragraph environment simply removes the first-line indent, following classical typography conventions. This subtle treatment is appropriate for the first paragraph after major headings.
\end{openingparagraph}

Subsequent paragraphs return to the standard indentation. This creates a clean, professional appearance without any decorative elements.

\section{Recommendations for Academic Use}

\begin{itemize}
\item Use special treatments sparingly—let the content speak
\item Prefer white space over ornamental breaks
\item Avoid drop caps unless specifically appropriate
\item Maintain consistency throughout the document
\item Follow journal or institutional guidelines
\end{itemize}

\section{Conclusion}

The most effective academic typography is often the most restrained. These minimal enhancements provide subtle improvements to readability and hierarchy while maintaining the professional appearance expected in scholarly publications.

\end{document}