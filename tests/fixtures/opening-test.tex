\documentclass[11pt]{article}
\usepackage{../../paper/paperstyle}
\usepackage{lipsum}

\begin{document}

\section{Introduction with Academic Drop Cap}

\academicdropcap{T}{his demonstrates} a conservative drop cap suitable for academic articles. The drop cap is modest in size, uses the body text color, and maintains professional appearance while adding subtle visual interest to major section openings.

\lipsum[1]

\section{First Line Small Caps Treatment}

\sectionopening{This first line appears in small caps with subtle tracking,} providing an elegant opening to sections without being ostentatious. The treatment is understated yet effective in creating visual hierarchy.

\lipsum[2]

\sectionbreak

\section{Section Break Demonstrations}

The three-asterisk break above is the most common in academic publishing. It provides clear visual separation without being decorative.

\lipsum[3]

\thinrulebreak

\subsection{Thin Rule Break}

The thin rule above offers a more subtle transition, suitable for subsection breaks or minor topic shifts within a larger section.

\lipsum[4]

\spacebreak

\subsection{Space Break}

The simplest break is just additional white space, as shown above. This is the most conservative option.

\section{Opening Paragraph Environment}

\begin{openingparagraph}
This paragraph uses the opening paragraph environment, which removes the first-line indent and can be used for the first paragraph after a major heading. It maintains the baseline grid while providing subtle differentiation.
\end{openingparagraph}

Regular paragraphs that follow return to the standard indentation pattern. \lipsum[5]

\section{Combined Treatments}

\begin{openingparagraph}
\firstlinesc{Combining the first line small caps} with the opening paragraph environment creates a sophisticated section opening. This approach is particularly effective for major divisions in the document.
\end{openingparagraph}

\lipsum[6]

\section{Abstract Opening Example}

\abstractopening

\begin{openingparagraph}
The abstract opening command provides a centered, small caps "ABSTRACT" label with appropriate spacing. This follows standard academic conventions while maintaining typographic refinement.
\end{openingparagraph}

\section{Conservative Drop Cap Guidelines}

\academicdropcap{W}{hen using drop caps} in academic articles, restraint is key:

\begin{itemize}
\item Use sparingly—perhaps only for the introduction or major sections
\item Maintain body text color for professionalism
\item Keep size modest (2 lines maximum)
\item Ensure the following text flows naturally
\item Consider the journal's style guidelines
\end{itemize}

\section{Professional Typography in Practice}

\sectionopening{Academic typography should enhance readability} without drawing attention to itself. These conservative treatments provide:

\begin{enumerate}
\item Clear visual hierarchy
\item Professional appearance
\item Enhanced readability
\item Subtle sophistication
\item Grid-perfect implementation
\end{enumerate}

\lipsum[7]

\majorsectionspace

\section{Conclusion}

These section opening enhancements maintain the professional standards expected in academic publishing while adding subtle typographic refinement. The key is restraint—every enhancement serves the content rather than decorating it.

\end{document}